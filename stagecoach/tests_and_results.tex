\documentclass[12pt]{article}
\usepackage[margin=1in]{geometry}
\usepackage{graphicx,amsmath,amssymb,blkarray,setspace,bm,gensymb}
\usepackage[authoryear,sectionbib,sort]{natbib}
\usepackage{lineno}
\usepackage[dvipsnames]{xcolor}
\usepackage[subtle]{savetrees}
\usepackage[compact]{titlesec} 
\usepackage[font={small,singlespacing},labelfont=bf,skip=2pt]{caption}
\usepackage{authblk,float}

\usepackage[utf8]{inputenc}
\usepackage[sc]{mathpazo} %Like Palatino with extensive math support

\usepackage{enumitem}
\setlist{topsep=-0.5em,itemsep=-0.2em,leftmargin=0.75cm}

\usepackage[formats]{listings}
\usepackage{color}
\definecolor{mygreen}{rgb}{0.1,0.5,0.1}
\definecolor{mygray}{rgb}{0.5,0.5,0.5}
\definecolor{mymauve}{rgb}{0.58,0,0.82}
\definecolor{mygrey}{rgb}{0.3,0.3,0.1}
\lstset{
  language=R,
  otherkeywords={data.frame},
  basicstyle=\normalsize\ttfamily, 
  commentstyle=\normalsize\ttfamily,
  keywordstyle=\normalsize\ttfamily,
  stringstyle=\color{mymauve}, 
  commentstyle=\color{mygreen},
  keywordstyle=\color{blue},
  showstringspaces=false, xleftmargin=2.5ex,
  columns=flexible,
  literate={~}{{$\sim \; \; $}}1,
  alsodigit={\.,\_},
  deletekeywords={on,by,data,R,Q,mean,var,sd,log,family,na,options,q,weights,effects,matrix,nrow,ncol,wt,fix,distance},
}
\lstset{escapeinside={(*}{*)}} 


%%%%%%%%%%%%%%%%%%%%%%%%%%%%%%%%%%%%%%%%%%%%%%%%%%%%%% 
%%% Our macros for frequently used symbols and such 
%%%%%%%%%%%%%%%%%%%%%%%%%%%%%%%%%%%%%%%%%%%%%%%%%%%%%% 
\DeclareMathOperator{\Ex}{\mathbb{E}}
\DeclareMathOperator{\var}{\textit{Var}}
\DeclareMathOperator{\cov}{\textit{Cov}}
\DeclareMathOperator{\vc}{\mbox{vec}}

\newcommand{\sk}{\widetilde{\mu}_3}

\newcommand{\be}{\begin{equation}}
\newcommand{\ee}{\end{equation}}


%%%%%%%%%%%%%%%%%%%%%%%%%%%%%%%%%%%%%%%%%%%%% 
%%% Just for commenting
%%%%%%%%%%%%%%%%%%%%%%%%%%%%%%%%%%%%%%%%%%%%
\newcommand{\new}{\textcolor{red}}
\newcommand{\comment}{\textcolor{blue}}

%%%%%%%%%%%%%%%%%%%%%%%%%%%%%%%%%%%%%%%%%%%%% 
%%% Some formatting control 
%%%%%%%%%%%%%%%%%%%%%%%%%%%%%%%%%%%%%%%%%%%% 
\renewcommand{\floatpagefraction}{0.9}
\renewcommand{\topfraction}{0.9}
\renewcommand{\textfraction}{0.05}
\sloppy
\clubpenalty = 1005
\widowpenalty = 1000

\title{Testing the \texttt{stageCoach} functions} 

\author{Steve Ellner} 


\begin{document} 

\maketitle 

\begin{figure}[h!]
\centering
\includegraphics[width=0.8\textwidth]{LifeCycle.pdf}
\caption{Life cycle used for testing the \texttt{stageCoach} functions. When one arrow exits
a compartment, the transition probability is 1. When multiple arrows exit a compartment, they are 
equally likely and their total probability is 1. Thus, all individuals eventually reach state 7 and then die. 
Only compartment state 6 has positive fecundity, with a 50\% chance of breeding.}  
\label{fig:testCycle}
\end{figure} 

The life cycle diagrammed in Figure \ref{fig:testCycle} is used to do some basic tests of the 
R translations of functions from \texttt{stageCoach}. Calculations are done in the script 
\texttt{testing\_conditional\_time\_funs.R}. 

\begin{itemize} 
\item 
\begin{verbatim} 
mean_lifespan(M7); 
[1] 4.666667 3.666667 3.000000 3.000000 2.000000 2.000000 1.000000
\end{verbatim} 
The function results for stages 3,5,6 and 7 are clearly right. Stage 4 has equal odds of lifespan 2 ($4 \to 7$) and
lifespan 4 ($4 \to 3 \to 6 \to 7$) so lifespan=3 is correct. The mean lifespan for stage 2 is one more
than the average for 3, 4, and 5, $1 + 8/3 = 3\frac{2}{3}$ and the mean lifespan for stage 1 is one more
than that -- both correct. 

\item 
\begin{verbatim} 
var_lifespan(M7); 
[1] 0.5555556 0.5555556 0.0000000 1.0000000 0.0000000 0.0000000 0.0000000
\end{verbatim} 
From stages 3, 5, 6 and 7 the route to the crypt is certain, so the function results are correct.  
Stage 4 has equal odds of lifespans 2 and 4, so the function result (variance=1) is correct. 
Stages 1 and 2 clearly have equal variance. From stage 2, $2 \to 3$ gives lifespan 4, $2 \to 5$ 
gives lifespan 3 --- each of these has probability 1/3. $2 \to 4$ gives equal odds of lifespans 
3 and 5, so these have probability 1/6 each. The function gives the correct value for the 
variance of this probability distribution. 







\end{itemize} 


 

\end{document} 
